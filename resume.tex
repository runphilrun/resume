%% RESUME - Philip Linden
\documentclass{resume}
\newcolumntype{Y}{>{\centering\arraybackslash}X} % centered and autosized cells in table
\begin{document}

\begin{tabularx}{\textwidth}{@{}llYr@{}}
\faGlobe & \href{https://runphilrun.github.io/}{runphilrun.github.io} & & pjl7651@rit.edu \\
\faLinkedin & \href{https://www.linkedin.com/in/philiplinden/}{philiplinden} &
\name{Philip J. Linden} & 440 N Winchester Blvd. Apt 128 \\
\faAngellist & \href{https://angel.co/philip-linden}{philip-linden} & & Santa Clara, CA 95050 \\
\faGithub & \href{https://github.com/runphilrun/}{runphilrun} & & (585) 690-7067 \\ 
\end{tabularx}

\section{Professional Summary}
I am a recent graduate who is passionate about the design and analysis of
aviation and space systems, including but not limited to satellites, human
spaceflight, spacecraft and aircraft structures, propulsion, mechanisms,
imaging, and controls. I am relentlessly curious, a strong visionary, and
optomistic about the future of technology and humankind.

\section{Degree}
\headerwithlabel{Rochester Institute of Technology}{Rochester, NY}{Aug 2012 --
  May 2017}
\begin{tabular}{ll}
Bachelor of Science in Mechanical Engineering -- Aerospace Option & {\bf GPA: } 3.5 \\
Master of Engineering in Mechanical Engineering (Dual Degree) & {\bf GPA: } 3.3 \\
\end{tabular}

\section{Graduate Paper}
\headerwithlabel{Cosmic Dawn Intensity Mapper System-Level Design}{\small\href{https://github.com/runphilrun/CDIM-design}{github.com/runphilrun/CDIM-design}}{}
Contributed to a proposal for a Probe Class (\textasciitilde\$850M) NASA mission for a 1.5
meter space telescope intended to observe near-infrared light from the early
universe.
\begin{itemize}
  \item Compiled financial, mass, and power budgets for the optics, instruments,
    cryocooler \& spacecraft.
  \item Defined system-level design, generated representative CAD models and
    figures for the entire spacecraft.
\end{itemize}
  
\end{document}